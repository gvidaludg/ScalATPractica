\documentclass{article}
\usepackage[T1]{fontenc}
\usepackage[catalan]{babel}

\begin{document}

\title{Informe de la Pràctica de Minizinc}
\author{Guillem Vidal}
\maketitle

\section{Explicació del model}

Tinc dues variables principals:

\begin{itemize}
	\item[schedule] \texttt{array[days, stadiums] of var set of teams}
	\item[place] \texttt{array[teams, days] of var stadiums}
\end{itemize}

La primera contindrà el calendari final dels partits, limitant que la
cardinalitat dels seus conjunts hagi de ser o bé de 0, o bé de 2. I la segona
descriu a quin estadi juga un equip en un dia concret, cosa que ajuda a limitar
que els equips només puguin jugar un sol partit al dia.

\texttt{place} és com una variable auxiliar, ja que pren el seu valor únicament
de \texttt{schedule} i simplement ajuda a restringir encara més la diversitat de
valors que l'última pot prendre.

\subsection{Restriccions}

Amb aquesta restricció limito que el paràmetre d'entrada \texttt{fixes} només
pugui tenir conjunts de cardinalitats menors o igual a 2, a més de limitar que
la cardinalitat dels conjunts de \texttt{schdule} a 0 o 2:

\begin{verbatim}
constraint forall(d in days, s in stadiums)(
  count(t in fixes[d, s])(t >= 0) <= 2 /\
  (count(t in schedule[d, s])(t >= 0) == 2
  \/ count(t in schedule[d, s])(t >= 0) == 0)
);
\end{verbatim}


Prosseguim amb la inicialització\footnote{Dir inicialitzar en aquest context
declaratiu i per restriccions és bastant incorrecte, però crec que fa bé la
feina a l'hora d'explicar la intenció darrere la restricció.} de la variable
\texttt{schedule} mitjançant \texttt{fixes}, on l'únic que requereixo és que el
valor a la posició de \texttt{fixes} sigui un subconjunt del valor a la mateixa
posició de \texttt{schedule}:

\begin{verbatim}
constraint forall(d in days, s in stadiums)(
  schedule[d, s] union fixes[d, s] == schedule[d, s]
);
\end{verbatim}

Amb la següent restricció comparo els conjunts directament, per assegurar-me que
no hi ha combinacions d'equips repetides en el calendari. No he pogut fer servir
la funció predefinida \texttt{all\_different} aquí, perquè els conjunts buits sí
que s'han de poder repetir:

\begin{verbatim}
constraint forall(d1 in days, s1 in stadiums)(
  forall(d2 in days, s2 in stadiums)(
    (d1 == d2 /\ s1 == s2) \/
    schedule[d1, s1] == {} \/
    schedule[d1, s1] != schedule[d2, s2]
  )
);
\end{verbatim}

Ara comencem a veure la variable \texttt{place}, la qual inicialitzem amb els
continguts trobats a \texttt{schedule}:

\begin{verbatim}
constraint forall(d in days, s in stadiums)(
  forall(t in schedule[d, s])(
    place[t, d] == s
  )
);
\end{verbatim}

En tenir un sol possible valor per cada equip i dia, estem limitant que un equip
no pugui jugar dos cops o més el mateix dia.

Per acabar, tenim la restricció que ens permet assegurar que cap equip jugarà
dues vegades o més al mateix estadi:

\begin{verbatim}
constraint forall(t in teams)(
  all_different([place[t, d] | d in days])
);
\end{verbatim}

\section{Eficiència}

Aquestes proves s'han executat en un Lenovo ThinkPad E15 amb una CPU Intel Core
i7 de 8 nuclis i 16 GB de memòria RAM.

Totes les proves s'han realitzat amb el \texttt{solve satisfy}, perquè tan sols
retorni la primera solució trobada. Les altres optimitzacions per cada solució
no m'han sortit prou bé com per fer-hi gaires proves.

He utilitzat el nivell d'optimització \texttt{-O5}.

\begin{table}[ht]
	\centering
	\begin{tabular}{|c|c|}
		\hline
		\emph{Prova} & \emph{Temps (s)} \\
		\hline
		\texttt{t0} & 1 \\
		\texttt{t1fix} & 4 \\
		\texttt{t1} & 6 \\
		\texttt{t2fix} & 21 \\
		\texttt{t3fix} & 121 \\
		% \texttt{t2} & \\
		%\texttt{t3} & \\
		%\texttt{t4} & \\
		\hline
	\end{tabular}
	\caption{Temps que ha tardat cada prova en arribar a una solució.}
	\label{tab:performance}
\end{table}

\end{document}
